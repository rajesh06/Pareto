\documentclass[]{article} %scrartcl
\usepackage{amsmath}
\usepackage{amsfonts}
\usepackage{parskip}
\usepackage{commath}
\usepackage{ifthen}
\usepackage[toc,page,header]{appendix}
\usepackage{arydshln}
\usepackage{booktabs}
\usepackage{multirow}
\usepackage{color}
\usepackage{nicefrac}

\newcommand{\philbrick}{\textit{Philbrick}}

%opening
\title{The Single Parameter Pareto Revisited}
\author{Rajesh Sahasrabuddhe, FCAS, MAAA}

\begin{document}

\maketitle

\begin{abstract}
In one of his seminal works, Stephen W. Philbrick proposed an elegant solution to the complex problem of modeling claims amounts in excess layers. 

\end{abstract}

\section{Introduction}
In one of his seminal works, Stephen W. Philbrick\cite{Philbrick} proposed an elegant solution to the complex problem of modeling claims amounts in excess layers. That solution involved the use a Pareto Type~I distribution for claims above an excess threshold.\\

This paper expands on that work in several ways by providing the following:
\begin{description}
	\item[Section \ref{sec:theSPP}] In this section we reintroduce the SPP, provide a consistent notational framework, and describes a test to determine when it appropriate to model claims using the SPP.
	\item[Section 3] discusses the Pareto parameter including a discussion of parameter values less than 1, the bias in the maximum likelihood estimator.
	\item [Section 3] We discuss the trend
	\item[Section 2] presents an Errata to \philbrick
	\item[The Appendix] A review of the various formul\ae~including supporting derivation
\end{description}

\section{The \emph{Single Parameter} Pareto}\label{sec:theSPP}
Philbrick's Single Parameter Pareto (SPP) is a special case of the Pareto Type~I distribution which has the following cumulative distribution function:
\begin{equation}
F(x) = 1 - \left(\frac{k}{x} \right)^a\label{eqn:Pareto1}
\end{equation}
Many readers are often confused by the reference to the \emph{single parameter}. After all, in Section~III Philbrick initially presents the Pareto with two parameters (Equation \ref{eqn:Pareto1}), $k$ and $a$, and then later adds that claims should be ``normalized'' by dividing by the ``selected lower bound.''

This presentation leaves many readers not understanding how the lower bound,~$k$,  ``lost'' its parameter status. Philbrick explains that this is because:
\begin{quote}
	Although there may be situations where this value must be estimated, in virtually all insurance applications this value will be selected in advance. (Section III)
\end{quote}
This explanation often leaves readers less than satisfied.

I offer the alternative that users of the SPP model should consider the process of normalizing the claims to be a transformation of the data rather than the application of a parameter. An analogous transformation occurs when we take $\log$s. When we do that, we do not consider the base of the logarithm to be a parameter. Similarly, we will not consider the lower bound to be a parameter.

To improve clarity of this concept, we present the following:

\begin{table}[h!]
	\centering
	\begin{tabular}[h]{ccc}
		\toprule
		\multirow{3}{*}{Random Variable}& Observed & Normalized\\
		& Claim & Claims\\
		& Amount & Amount \\ \midrule
		Symbol & $\mathbf{Y}$ & $\mathbf{X}$\\ \midrule
		\multirow{2}{*}{Transformation} & & $\mathbf{X}$  =  $g(\mathbf{Y})$\\
		& & $g(\mathbf{Y}) =   \mathbf{Y}/\text{lower bound}$  \\ \midrule
		Domain & [\text{lower bound}, $\infty$] & [1, $\infty$]\\ \midrule
		Density & $a \dfrac{k^a}{y^{a+1}}$ & $qx^{-(q+1)}$\\ \midrule
		Parameters & $k>0$ (scale); $a>0$ (shape)& $q>0$ (shape)\\ 
	\bottomrule
	\end{tabular}
	\caption{The Pareto Type I and the SPP}\label{tbl:PIvSpp}			
\end{table}
We can now work with model forms in the space of $\mathbf{X}$ and then use $g^{-1}$ to transform back into the space of $\mathbf{Y}$. We can also now present the density and distribution functions.

\begin{align}
		f(x) & =  qx^{-(q+1)}\label{SPPf}\\
		F(x) & =  1 - x^{-q}\label{SPPF}
\end{align}

\subsection{Tail weight and the SPP}
The appropriate application of the SPP is in the modeling of claims in the tail of a distribution. The SPP is generally used where the tail is said to be ``thick'' or ``heavy.'' We can define the weight of a tail in terms of how quickly the density function approaches $0$ (light-tailed distributions approach $0$ more quickly) or how quickly the distribution function approach $1$ (light-tailed distributions approach $1$ more quickly). We will use the distribution function since the domain is more easily understood and comparable between distributions. 

Effectively we measure tail weight as the slope of the distribution function and 

For a sample $x_1$ \ldots $x_n$ we can compare the tail function of a distribution to the empirical tail weight function.
\begin{table}[h!]
	\centering
	\begin{tabular}[h]{cc}
	\toprule
	Model tail-weight & $TW_m(i)$ = 
	$\dfrac{
		\nicefrac{1}{n} %F(x_{(i+1)}) - {F(x_{(i)})}
	}{
		F^{-1}(\frac{i+1}{n}) %{x_{(i+1)}} 
		- F^{-1}(\nicefrac{i}{n}) %{x_{(i)}}
	}$\\
	\midrule
	Empirical tail-weight & $TW_e(i)$ = 
	$\dfrac{
		\nicefrac{1}{n}
	}{
		{x_{(i+1)}}-{x_{(i)}}
	}$\\
	\bottomrule
	\end{tabular}
\end{table}

Where we use $x_{(i)}$ to indicate the $i$th order statistic of the sample and $F^{-1}$ to refer to the inverse of the the cumulative distribution function.

Of course, we still have to identify where we measure tail weight. Rather than identifying a single point (or percentile), it is more effective to plot the functions.





We can measure tail weight based on the change in claim size per unit change in the distribution percentile. Specifically, we calculate the following measure of tail weight:

In this section, we compare the tail weight of the Pareto to that of the lognormal distribution.

We also note that the mode of the SPP is at its minimum value of $x = 1$. Conversely, for the lognormal, the mode is right of its minimum value of  of $x = 0$.

In  Appendix \ref{ParetoF}, we provide the derivation of the distribution function (Equation (\ref{SPPF})).

\subsection{Other Pareto Forms}
In Appendix \ref{ParetoInventory}, we provide an inventory of Pareto distributions including the SPP presented in Equations (\ref{SPPf}) and (\ref{SPPF}). 

\section{Actuarial Formul\ae}
In general, we leave formula derivation to the Appendices of this paper. In Appendix \ref{derive:ParetoEV}, we present the derivation of the expected value:

\begin{equation}
\mathbb{E}[X]  = 
\begin{cases}
\dfrac{q}{q-1},& \text{if } q > 1\\
\text{undefined},& \text{if } q \leq 1\\         
\end{cases}
\end{equation}

\section{The Pareto Parameter}


\end{document}
