\newpage
\appendix
\appendixpage
% How to insert here a TOC concerning only the appendix sections ????

\section{An Inventory of Pareto Distributions}\label{ParetoInventory}
Appendix first content.

\section{Derivation of Forumul\ae}

\subsection{Expected Values}\label{sec:ParetoEV}
\begin{align}
\mathbb{E}[X] & = \int_{1}^{\infty}x f(x) \dif x \nonumber\\
& = \int_{1}^{\infty}xqx^{-(q+1)} \dif x\nonumber\\
& = q \int_{1}^{\infty}x^{-q} \dif x\nonumber\\
& = q \frac{1}{-q+1} x^{-q+1} \Big|_{1}^{\infty}\nonumber\\
& = \frac{q}{1-q} x^{-q+1} \Big|_{1}^{\infty}\nonumber\\
\mathbb{E}[X] 	& = \frac{q}{1-q} \dfrac{1}{x^{q-1}} \Big|_{1}^{\infty}\label{eqn:ParetoEV}
\end{align}
We can see that for $x = 1$ (the lower limit of integration) equation (\ref{EV}) evaluates to $\dfrac{q}{1-q}$. However for $x = \infty$ (the upper limit of integration), we have the following\footnote{In the limit as $x\to\infty$, the expression evaluates to  $-\dfrac{q}{q+1}$. However evaluated at $\infty$, the expression is undefined.}:

\[
\frac{q}{1-q} \dfrac{1}{x^{q-1}} = 
\begin{cases}
0,& \text{if } q > 1\\
\text{undefined},& \text{if }q = 1\\         
\infty, & \text{if }q < 1
\end{cases}
\]

and therefore we have: 
\[
\mathbb{E}[X]  = 
\begin{cases}
0 - \dfrac{q}{1-q},& \text{if } q > 1\\
\text{undefined},& \text{if } q = 1\\         
\infty, & \text{if }q < 1
\end{cases}
\]
or more simply:

\begin{equation}
\mathbb{E}[X]  = 
\begin{cases}
\dfrac{q}{q-1},& \text{if } q > 1\\
\text{undefined},& \text{if } q \leq 1\\         
\end{cases}
\end{equation}

\subsection{SPP Cumulative Distribution Function}\label{ParetoF}
\begin{align}
	F(x) &= \int_{1}^{x} f(x) \dif x \nonumber\\
		&= \int_{1}^{x}qx^{-(q+1)} \dif x\nonumber\\
		&= q \int_{1}^{x}x^{-(q+1)} \dif x\nonumber\\
		&= q \frac{1}{-(q+1)+1} x^{-q} \Big|_{1}^{x}\nonumber\\
		&=  q \frac{1}{-q} x^{-q} \Big|_{1}^{x}\nonumber\\
		&=  - x^{-q}\Big|_{1}^{x}\nonumber\\
		&= -x^{-q} - (-1^{-q})\nonumber\\
	 F(x) &=  1 - x^{-q}
\end{align}

\subsection{Limited Expected Value}\label{sec:ParetoLEV}

We can derive the expected values as:
\begin{align}
\mathbb{E}[X] & = \int_{1}^{\infty}x f(x) \dif x \nonumber\\
& = \int_{1}^{\infty}xqx^{-(q+1)} \dif x\nonumber\\
& = q \int_{1}^{\infty}x^{-q} \dif x\nonumber\\
& = q \frac{1}{-q+1} x^{-q+1} \Big|_{1}^{\infty}\nonumber\\
& = \frac{q}{1-q} x^{-q+1} \Big|_{1}^{\infty}\nonumber\\
\mathbb{E}[X] 	& = \frac{q}{1-q} \dfrac{1}{x^{q-1}} \Big|_{1}^{\infty}\label{EV}
\end{align}
We can see that for $x = 1$ (the lower limit of integration) equation (\ref{EV}) evaluates to $\dfrac{q}{1-q}$. However for $x = \infty$, we have the following\footnote{In the limit as $x\to\infty$, the expression evaluates to  $-\dfrac{q}{q+1}$. However evaluated at $\infty$, the expression is undefined.}:

\[
\frac{q}{1-q} \dfrac{1}{x^{q-1}} = 
\begin{cases}
0,& \text{if } q > 1\\
\text{undefined},& \text{if } q = 1\\         
\infty, & \text{if }q < 1
\end{cases}
\]

and therefore we have: 
\[
\mathbb{E}[X]  = 
\begin{cases}
0 - \dfrac{q}{1-q},& \text{if } q > 1\\
\text{undefined},& \text{if } q = 1\\         
\infty, & \text{if }q < 1
\end{cases}
\]
or more simply:

\begin{equation}
\mathbb{E}[X]  = 
\begin{cases}
\dfrac{q}{q-1},& \text{if } q > 1\\
\text{undefined},& \text{if } q \leq 1\\         
\end{cases}
\end{equation}

\subsection{SPP Limited Expected Value}\label{ParetoLEV}

The limited expected value is calculated as:
\begin{align}
	\mathbb{E}[X;b] 	&= \frac{q}{1-q} \dfrac{1}{x^{q-1}} \Big|_{1}^{b} + b  (1-F(b))\nonumber\\
 	&=  \frac{q}{1-q} \left[ \dfrac{1}{b^{q-1}} - \dfrac{1}{1^{q-1}}\right]+  b  \left[1-(1-b^{-q})\right]\nonumber\\
 	&=  \frac{q}{1-q} \left[ \dfrac{1}{b^{q-1}} - 1\right]+  b  \left[b^{-q}\right]\nonumber\\
 	&=  \frac{q}{q-1} \left[1 - \dfrac{1}{b^{q-1}}\right]+ b^{1-q}\nonumber\\
 	&=  \frac{q}{q-1} \left[1 - b^{1-q}\right]+b^{1-q}\\ %This is the formula with E[X] 
 	&=  \frac{1}{q-1} \left[q - qb^{1-q} + (q-1)b^{1-q}\right]\nonumber\\
 	&=  \frac{1}{q-1} \left[q - b^{1-q}\right]\nonumber\\
 	&=	\frac{q - b^{1-q}}{q-1}\label{eqn:ParetoLEV}
\end{align}
\newpage

\subsection{Maximum Likelihood Estimator for Parameter}
 The negative log-likelihood ($NLL$) function given data $D = x_1 \ldots x_n$ is defined as:
 \begin{align*}
 L(q) &= \prod_{i=1}^{n} f{x_i}\\
 \text{NLL} &= -\sum_{i=1}^{n} \ln(f{x_i}) \\
 \text{NLL} &= -\sum_{i=1}^{n}\ln(qx_i^{-(q+1)})\\
 \text{NLL} &= -\sum_{i=1}^{n}\left[\ln{q} + \ln{x_i^{-(q+1)}}\right]\\
 \text{NLL} &= -\sum_{i=1}^{n}\left[\ln{q} - (q+1)\ln{x_i}\right]\\
 \text{NLL} &= -n\ln{q} + \sum_{i=1}^{n}(q+1)\ln{x_i}\\
 \text{NLL} &= -n\ln{q} + (q+1)\sum_{i=1}^{n}\ln{x_i}\\
  \end{align*}
 We can calculate the MLE of $q$ by taking partial derivatives and setting equal to $0$.
  \begin{align*}
 0 &= \frac{\partial}{\partial q}\left[ 
 -n\ln{q} + 
 (q+1)\sum_{i=1}^{n}\ln{x_i}
 \right] \\
 0 &= \frac{\partial}{\partial q}\left[ \sum_{i=1}^{n}\ln(q) + \sum_{i=1}^{n} -(q+1)\ln{x_i}\right] \\
 0 &= n - \frac{\partial}{\partial q}\sum_{i=1}^{n} q\ln{x} - \frac{\partial}{\partial q} \sum_{i=1}^{n} \ln{x} 
 \end{align*}