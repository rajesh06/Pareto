\documentclass[]{article} %scrartcl
\usepackage{amsmath}
\usepackage{amsfonts}
\usepackage{parskip}
\usepackage{commath}
\usepackage{ifthen}
\usepackage[toc,page,header]{appendix}
\usepackage{arydshln}


%opening
\title{The Single Parameter Pareto Revisited}
\author{Rajesh Sahasrabuddhe, FCAS, MAAA}

\begin{document}

\maketitle

\begin{abstract}
In one of his seminal works, Stephen W. Philbrick proposed an elegant solution to the complex problem of modeling claims amounts in excess layers. 

\end{abstract}

\section{Introduction}
In one of his seminal works, Stephen W. Philbrick\cite{Philbrick} proposed an elegant solution to the complex problem of modeling claims amounts in excess layers. That solution involved the use a Pareto Type I distribution for claims above an excess threshold.\\

This paper expands on that work in several ways through the following:
\begin{enumerate}
	\item A consistent notational framework
	\item A review of the various formulae including supporting derivation in the Appendices
	\item Parameter values less than or equal to 1
	\item Consideration of maximum probable loss using order statistics
	\item Quantification of parameter uncertainty using Bayesian Markov chain Monte Carlo (MCMC) methods (expanding on the work of Meyers and Reichle \& Yonkunis)
	\item Frequency and severity trends
	\item Dealing with claims less than the threshold (expanding on the work of Fackler)
\end{enumerate}
\section{Preliminaries}\label{Prelims}
\subsection{The \emph{Single Parameter} Pareto}
Many reviewers are often confused by the reference to the \emph{single parameter}. After all \emph{Philbrick} in Section III:
\begin{itemize}
\item Initially presents a Pareto with two parameters, $k$ and $a$.
\item Later adds that claims should be ``normalized'' by dividing by the ``selected lower bound''
\end{itemize}
This presentation leaves many readers not understanding how the lower bound,~$k$,  lost its parameter status. 

Users of the SPP model should consider the process of normalizing the claims to be a transformation of the data rather than the application of a parameter. An analogous transformation occurs when we take $\log$s. When we do that, we do not consider the base of the logarithm to be a parameter. Similarly, we will not consider the lower bound to be a parameter.

To improve clarity of this concept, we present the following notation:
\begin{eqnarray}
	\begin{array}{ccl}
		\mathbf{Y} & = & \text{observed claim random variable} \\ 
		\mathbf{Y}& \in &  [\text{lower bound}, \infty]\\
		\cline{1-3}
		\mathbf{X} & =  & g(\mathbf{Y})\\
		g(\mathbf{Y}) & =  & \mathbf{Y}/\text{lower bound}\\
		\mathbf{X}& \in &  [1, \infty]\\
	\end{array} 
\end{eqnarray}
We can now work with model forms in the space of $\mathbf{X}$ and then use $g^{-1}$ to transform back into the space of $\mathbf{Y}$. We can also now present the density and distribution functions.

\begin{align}
		f(x) & =  qx^{-(q+1)}\label{SPPf}\\
		F(x) & =  1 - x^{-q}\label{SPPF}
\end{align}

In Appendix \ref{ParetoInventory}, we provide an inventory of Pareto distributions including the SPP presented in Equations (\ref{SPPf}) and (\ref{SPPF}). In  Appendix \ref{ParetoF}, we provide the derivation of the distribution function (Equation (\ref{SPPF})).

\subsection{Actuarial Formul\ae}
\begin{align}
	\mathbb{E}[X] & = \int_{1}^{\infty}x f(x) \dif x \nonumber\\
		& = \int_{1}^{\infty}xqx^{-(q+1)} \dif x\nonumber\\
		& = q \int_{1}^{\infty}x^{-q} \dif x\nonumber\\
		& = q \frac{1}{-q+1} x^{-q+1} \Big|_{1}^{\infty}\nonumber\\
		& = \frac{q}{1-q} x^{-q+1} \Big|_{1}^{\infty}\nonumber\\
	\mathbb{E}[X] 	& = \frac{q}{1-q} \dfrac{1}{x^{q-1}} \Big|_{1}^{\infty}\label{EV}
\end{align}
We can see that for $x = 1$ (the lower limit of integration) equation (\ref{EV}) evaluates to $\dfrac{q}{1-q}$. However for $x = \infty$, we have the following\footnote{In the limit as $x\to\infty$, the expression evaluates to  $-\frac{q}{q+1}$. However evaluated at $\infty$, the expression is undefined.}:

\[
    \frac{q}{1-q} \dfrac{1}{x^{q-1}} = 
\begin{cases}
    0,& \text{if } q > 1\\
    \text{undefined},& \text{if } q = 1\\         
    \infty, & \text{if }q < 1
\end{cases}
\]

and therefore we have: 
\[
    \mathbb{E}[X]  = 
\begin{cases}
    0 - \dfrac{q}{1-q},& \text{if } q > 1\\
    \text{undefined},& \text{if } q = 1\\         
    \infty, & \text{if }q < 1
\end{cases}
\]
or more simply:
\[
    \mathbb{E}[X]  = 
\begin{cases}
    \dfrac{q}{q-1},& \text{if } q > 1\\
    \text{undefined},& \text{if } q \leq 1\\         
\end{cases}
\]
%For the remainder of Section \ref{Prelims}, we will restrict ourselves to the $q>1$ case. 

We can use Equation (\ref{EV}), to calculate the limited expected value through $b$\footnote{Philbrick used $b$ to refer to both the ``lower bound" and the policy limit. We will not do that in this paper primarily for clarity as using a variable to represent the lower bound implied at least the possibility that the lower bound was a parameter. Conveniently, it also allows us  to use the traditional policy notation as attaching at $a$ through limit $b$ with the resulting layer width equal to $b - a$.} as:
\begin{align}
	\mathbb{E}[X;b] = 
\end{align}
\newpage
\appendix
\appendixpage
% How to insert here a TOC concerning only the appendix sections ????

\section{An Inventory of Pareto Distributions}\label{ParetoInventory}
Appendix first content.
\section{Derivation of Forumul\ae}
\subsection{SPP Cumulative Distribution Function}\label{ParetoF}
\begin{align}
	F(x) &= \int_{1}^{x} f(x) \dif x \nonumber\\
		&= \int_{1}^{x}qx^{-(q+1)} \dif x\nonumber\\
		&= q \int_{1}^{x}x^{-(q+1)} \dif x\nonumber\\
		&= q \frac{1}{-(q+1)+1} x^{-q} \Big|_{1}^{x}\nonumber\\
		&=  q \frac{1}{-q} x^{-q} \Big|_{1}^{x}\nonumber\\
		&=  - x^{-q}\Big|_{1}^{x}\nonumber\\
		&= -x^{-q} - (-1^{-q})\nonumber\\
	 F(x) &=  1 - x^{-q}
\end{align}
\subsection{Limited Expected Value}
For $q > 1$, the limited expected value is calculated as:
\begin{align}
	\mathbb{E}[X;b] 	&= \frac{q}{1-q} \dfrac{1}{x^{q-1}} \Big|_{1}^{b} + b  (1-F(b))\nonumber\\
 	&=  \frac{q}{1-q} \left[ \dfrac{1}{b^{q-1}} - \dfrac{1}{1^{q-1}}\right]+  b  \left[1-(1-b^{-q})\right]\nonumber\\
 	&=  \frac{q}{1-q} \left[ \dfrac{1}{b^{q-1}} - 1\right]+  b  \left[b^{-q}\right]\nonumber\\
 	&=  \frac{q}{q-1} \left[1 - \dfrac{1}{b^{q-1}}\right]+ b^{1-q}\nonumber\\
 	&=  \frac{q}{q-1} \left[1 - b^{1-q}\right]+b^{1-q}\nonumber\\
 	&=  \frac{q}{q-1} \left[1 - b^{1-q}\right]+ b^{1-q}\nonumber\\
 	&=  \frac{q}{q-1} \left[1 - b^{1-q}\right]+b^{1-q}\nonumber\\
 	&=  \frac{q}{q-1} \left[1 - b^{1-q}\right]+ b^{1-q}\\
 	&=	\frac{q}{q-1} \left[1 - b^{1-q} + \frac{q}{q-1} b^{1-q}\right]\nonumber\\
 %	&=	\frac{q}{q-1} \left[1 + (\frac{q}{q-1}-1) b^{1-q}\right]\nonumber\\	
 %	&=	\frac{q}{q-1} \left[1 + (\frac{1}{q-1}) b^{1-q}\right]\nonumber\\
 %	&=	\frac{q}{q-1} \left[1 + \frac{b^{1-q}}{q-1} \right]\nonumber\\
 %	&=	\frac{q}{q-1} \left[\frac{q-1+b^{1-q}}{q-1} \right]\nonumber\\	
 \end{align}
\bibliographystyle{plain}
\bibliography{MyBib}
\end{document}
