\documentclass[]{article} %scrartcl
\usepackage{amsmath}
\usepackage{parskip}
\usepackage{commath}
\usepackage{ifthen}


%opening
\title{The Single Parameter Pareto Revisited}
\author{Rajesh Sahasrabuddhe, FCAS, MAAA}

\begin{document}

\maketitle

\begin{abstract}
In one of his seminal works, Stephen W. Philbrick proposed an elegant solution to the complex problem of modeling claims amounts in excess layers. 

\end{abstract}

\section{Introduction}
In one of his seminal works, Stephen W. Philbrick\cite{Philbrick} proposed an elegant solution to the complex problem of modeling claims amounts in excess layers. That solution involved the use a Pareto Type I distribution for claims above an excess threshold.\\

This paper expands on that work in several ways through the following:
\begin{enumerate}
\item A consistent notational framework
\item A review of the various formulae including supporting derivation in the Appendices
\item Consideration of maximum probable loss using order statistics
\item Quantification of parameter uncertainty using Bayesian Markov chain Monte Carlo (MCMC) methods (expanding on the work of Meyers and Reichle \& Yonkunis)
\item Frequency and severity trends
\item Dealing with claims less than the threshold (expanding on the work of Fackler)
\end{enumerate}
\section{Notational Framework}
\subsection{The \emph{Single Parameter} Pareto}
Many reviewers are often confused by the reference to the \emph{single parameter}. After all \emph{Philbrick} In Section III:
\begin{itemize}
\item Initially presents a Pareto with two parameters, $k$ and $a$.
\item Later adds that claims should be ``normalized'' by dividing by the ``selected lower bound'' - 
\end{itemize}
This presentation leaves many readers not understanding how the lower bound, $k$,  lost its parameter status. 

Users of the SPP model should consider the process of normalizing the claims to be a transformation of the data rather than the application of a parameter. An analogous transformation occurs when we take $\log$s we do not consider the base of the logarithm to be a parameter.

To improve clarity of this concept, we present the following notation:
\begin{eqnarray}
	\begin{array}{ccl}
		\mathbf{Y} & = & \text{observed claim random variable} \\ 
		\mathbf{Y}& \in &  [\text{lower bound}, \infty]\\	
		\mathbf{X} & =  & g(\mathbf{Y})\\
		g(\mathbf{Y}) & =  & \mathbf{Y}/\text{lower bound}\\
		\mathbf{X}& \in &  [1, \infty]\\
	\end{array} 
\end{eqnarray}
We can now work with model forms in the space of $\mathbf{X}$ and then use $g^{-1}$ to transform back into the space of $\mathbf{Y}$. We can also now present the density and distribution functions.

\begin{align}
		f(x) & =  qx^{-(q+1)}\\
		F(x) & =  1 - x^{-q}
\end{align}

\begin{eqnarray}
\begin{array}{ccl}
	F(x) & = &\int_{1}^{x} f(x) \dif x \\	
		& = &\int_{1}^{x}qx^{-(q+1)} \dif x\\
		& = &q \int_{1}^{x}x^{-(q+1)} \dif x\\
		& = &q \frac{1}{-(q+1)+1} x^{-(q+1)+1} \bigg|_{1}^{x}\\
		& = &q \frac{1}{-q} x^{-q} \bigg|_{1}^{x}\\
		& = & - x^{-q}\bigg|_{1}^{x}\\
		& = & 1 - x^{-q}
\end{array}
\end{eqnarray}

\subsection{Actuarial Formul\ae}
\bibliographystyle{plain}
\bibliography{MyBib}
\end{document}
